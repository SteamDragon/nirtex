% второй раздел - файл part2.tex

\section{Название следующего раздела}

\subsection{Исходные коды}

Для отображения исходных кодов (см. пример ссылки на литературу \cite{Rrus1}) можно  воспользоваться окружением \verb|MyCode|:

\begin{MyCode}
require(ggplot2)
gi = ggplot(data = iris, aes(x=Sepal.Length,y=Petal.Length))
gi + geom_point(aes(color=Species))
\end{MyCode}


\subsection{Нумерованные листинги}

Для нумерации листингов и возможности ссылаться на них  предназначено окружение \verb|Program|.

\begin{Program}
\begin{MyCode}
#include <stdio.h>

int main(int argc, char **argv){
  printf("Hello, world");
  return 0;
}
\end{MyCode}
\caption{Код приветствия}
\end{Program}

Следует учесть, что объем кода в таком листинге должен быть небольшим --- меньше страницы, а нумерация появляется, если внутри окружения задано название листинга с помощью команды \verb|\caption|. 

Кроме того, положение кода выбирается системой \LaTeXe.
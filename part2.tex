% второй раздел - файл part2.tex

\section{Методики учета семейного бюджета}

\subsection{Где вести учет семейного бюджета}

\subsubsection{Тетрадь или амбарная книга}
Несомненно, что для вычислений, связанных с учетом личных финансов,
было бы удобно воспользоваться компьютером и вести в нем все записи, однако если такой возможности нет, то можно завести тетрадь или амбарную
книгу. В самом простом обобщенном случае рекомендуется разбить лист на
три графы:
\begin{table}[H]
\caption{Пример теблицы для учета семейного бюджета}
\label{tab:t1}
\begin{center}
\begin{tabular}{|r|p{5.5cm}|p{2.5cm}|}
\hline 
Доход & Расход & Итого \\ 
\hline 
 &  &  \\ 
\hline 
\end{tabular} 
\end{center}
\end{table}

Графы Расход и Доход будут отражать соответствующее движение
денег вашего кошелька, а графа Итого нужна для того, чтобы сверять цифры на бумаге с количеством денег в карманах. Как ни странно, они должны
совпадать.
Такой подход в целом приемлем для одного человека, он даже позволит
отследить и выявить необязательные расходы, которые впоследствии можно
уменьшить или вовсе убрать. Однако в таком виде о какой-либо наглядности
и систематизации говорить не приходится. Тем более в рамках рассмотрения
бюджета семьи. Ведь, как уже говорилось в предыдущей части, семейный
бюджет охватывает множество составляющих.
Для повышения наглядности и хоть какой-то систематизации доходов и
расходов, приведенную табличку необходимо разбавитьї дополнительными
колонками группируя разные виды расходов в соответствии реально имеющимся.
Например, в первую колонку можно записывать коммунальные платежи,
свет, интернет или аренду. Во второй колонке записывать лишь траты на
продукты питания, в третьей личные расходы, в четвертой расходы на развлечения и в пятой непредвиденные расходы.
Естественно, существующую таблицу нужно модернизировать под себя и
вероятно кто-то посчитает нужным добавить колонки по бытовой химии, уходу за кошкой, ребенком, родителями и т.д.
Эти расширения, в конце концов, приведут к тому, что таблица попросту
перестанет умещаться даже в амбарную книгу. И в этом случае на помощь
приходят компьютерные программные средства.

\subsubsection{Электронные таблицы}
Более продвинутый путь, приступить к ведению семейного бюджета при
помощи электронной таблицы (Excel, Google Docs и т.п.), где уже даже имеются основные формулы для анализа бюджета. По сути, вам остается лишь
выбрать и применить их к своим данным. Но и тут есть путь проще.
Дело в том, что на сегодняшний день существует множество специальных
шаблонов для электронных таблиц, в которых уже учтены некоторые наиболее популярные поля и необходимые для расчетов формулы.
\subsubsection{Специализированные программы}
Кроме шаблонов для табличных редакторов, в сети интернет предлагается
масса специальных программ для ведения учета и планирования семейного
бюджета.
Они позволяют автоматизировать большую часть работы, что значительно
упрощает процесс ведения домашних финансов.
Помимо того, эти программы, как правило, имеют массу вспомогательных функций, которые позволяют выявить слабые и сильные стороны вашего отношения с деньгами, помогут явно обратить внимание на, казалось бы,
очевидные, вещи, но почему-то не используемые в повседневной жизни. По
сути, программы для ведения семейного бюджета значительно облегчают и
помогают создать целостную картину наших взаимоотношений с финансами.

\part{Техническое задание на разработку программного обеспечения }

\section{Общие положения}
\subsection{Наименование программы}
Наименование программы: «Программа учёта персональных финансов Snipe Studio Budget Manager».
\subsection{Назначение и область применения}
Программа предназначена для учета домашних финансов и личных средств и не может быть применена для ведения бухгалтерии на предприятиях.
\subsection{Основания для разработки}
Данная программа разрабатывается в рамках постановки задачи от руководителя выпускной квалификационной работы для ИАТЭ НИЯУ МИФИ. 
\section{Требования к программе}
\subsection{Требования к функциональным характеристикам }
Программа должна обеспечивать возможность выполнения перечисленных ниже функций:
\begin{enumerate}
	\item Функции добавления записи в базу данных. Доход и расход должны быть выполнены с использованием отдельных кнопок
	\item Функции изменения записей в базе данных с возможностью изменить тип записи
	\item Функции удаления записей в базе данных без возможности их восстановления
	\item Возможности изменения места хранения лога приложения и базы данных через окно выбора директории
	\item Возможность изменения отображаемой валюты
	\item Возможность изменения локализации для следующих языков: \begin{enumerate}
		\item Русский
		\item Английский
		\item Голландский
		\item Немецкий
		\item Французский
	\end{enumerate}
	\item Возможность изменить уровень логирования вплоть до полного отключения записи логов
	\item Функцию полной очистки базы данных без возможности восстановления
	\item Функцию экспорта данных из программы в согласованом формате
	\item Функцию импорта данных в программу из файла, записанного в согласованном формате
	\item Возможность комфортной работы в полноэкранном режиме (при развороте приложения)
	\item Функцию работы с горячими клавишами	
\end{enumerate}

\subsection{Требования к организации входных данных}
Данные для работы программы предварительно не требуется организовывать. Исключение составляет только функция импорта. Формат экспортируемых и импортируемых данных:\\
\textit{1-ая строка: количество добавляемых записей\\
2-ая и последующие строки содержат список полей записи разделенных символом «;»\\
1-я позиция — порядковый номер записи, при импорте игнорируется.\\
2-я позиция — дата и время выполнения операции\\
3-я позиция — Сумма операции\\
4-я позиция — Коментарий к операции\\
5-я позиция — не используется\\
6-я позиция — Тип операции (1 — доход, 0 — расход)}

Тип файла для импорта должен соответствовать текстовому файлу (csv, txt, и.т.п.)\\

Файлы указанного формата должны размещаться (храниться) на локальных или съемных носителях, отформатированным согласно требованиям операционной системы.

\subsection{Требования к организации выходных данных}
См. Требования к организации входных данных\\
\subsection{Требования к временным характеристикам}
Требования к временным характеристикам к программе не предъявляются\\

\subsection{Требования к надежности}
\subsubsection{Требования к обеспечению надежного (устойчивого) функционирования программы}
Надежное (устойчивое) функционирование программы должно быть обеспечено выполнением Заказчиком совокупности операционно-технических мероприятий перечень которых приведен ниже:
\begin{enumerate}
	\item Организацией бесперебойного питания технических средств
\end{enumerate}

\subsubsection{Время восстановления после отказа}
Время восстановления после отказа, вызванного сбоем электропитания технических средств (иными внешними факторами), не фатальным сбоем (не крахом) операционной системы, не должно превышать 15 секунд без учета времени полной загрузки операционной системы.\\

Время восстановления после отказа, вызванного фатальным сбоем (крахом) операционной системы не должно превышать времени, требуемого на устранение неисправностей технических средств и переустановки програмных средств.\\
\subsubsection{Отказы из-за некорректных действий пользователя}
Отказы программы возможны в случае некорректных действий пользователя при взаимодействии с операционной системой. Во избежании возникновения отказов по указанной выше причине следует обеспечить работу конечного пользователя без предоставления ему административных привелегий\\
\section{Условия эксплуатации}
\subsubsection{Климатические условия эксплуатации}
Климатические условия эксплуатации, при которых должны обеспечиваться данные характеристики, должны удовлетворять требованиям, предъявляемым к техническим средствам в части условий их эксплуатации.\\
\section{Требования к видам обслуживания}
Программа не требует проведения каких либо видов обслуживания
\section{Требования к численности и квалификации персонала}
Минимальное количество персонала, требуемого для работы программы, должно составлять не менее 1 штатной единицы — конечный пользователь программы — оператор.\\

Конечный пользователь программы (оператор) должен обладать минимальными навыками работы с графическим пользовательским интерфейсом операционной системы.\\
\section{Требования к составу и параметрам технических средств.}
В состав технических средств должен входить IBM-совместимый персональный компьютер на базе процессоров архитектур x64 или x86 c размером свободной оперативной памяти не менее 50 Мб\\
\section{Требования к информационным структурам и методам решения}
Требования к информационным структурам (файлов) на входе и выходе, а также к методам решения не предъявляются\\

Исходные коды программы должны быть реализованы на языке C++ с расширением QT5. В качестве интегрированной среды разработки программы должна быть использована среда QTCreator.\\
\section{Требования к защите информации и программ}
Требования к защите информации и программ не предъявляется.\\
\section{Требования к маркировке и упаковке.}
Требования к маркировке и упаковке не предъявляется.\\
\section{Специальные требования}
Программа должна обеспечивать взаимодействие с пользователем (оператором) посредством графического интерфейса, согласно рекомендациям компании-производителя операционной системы.\\
\section{Технико-экономические показатели}
Ориентировочная экономическая эффективность не расчитывается\\
Предполагаемое использование программы в год до 2000 сеансов работы в год\\

\section{Стадии и этапы разработки}
\subsection{Стадии разработки}
Разработка должна производится в 2 этапа:
\begin{enumerate}
	\item Разработка технического задания
	\item Рабочее проектирование
\end{enumerate}

Этап внедрения не предполагается, т.к. система не предназначена для использования на предприятиях\\
\subsubsection{Этапы разработки}
На стадии разработки технического задания должен быть выполнен этап разработки, согласования и утверждения настоящего технического задания.\\

На стадии рабочего проектирования должны быть выполнены перечисленные ниже этапы работ:
\begin{enumerate}
	\item Разработка программы
	\item Испытание программы
\end{enumerate}

\subsection{Содержание работы по этапам}
На этапе разработки технического задания должны быть выполнены перечисленные ниже работы:
\begin{enumerate}
\item Постановка задач
\item Определение и уточнение требований к технической документации
\item Определение требований к программе
\item Определение стадий, этапов и сроков разработки программы и документации
\item выбор языка программирования
\item Согласование и утверждение технического задания
\end{enumerate}

На этапе разработки должно быть выполнено кодирование и отладка программы.\\

На этапе испытаний должны быть выполнено проведение приемо-передаточных испытаний и корректировка программы по результатам испытаний.\\

% !TeX spellcheck = de_DE
\part{Выбор инструментов и средств разработки}
\section{Обзор и выбор СУБД}
\subsection{СУБД MySQL}
MySQL — свободная реляционная система управления базами данных. Разработку и поддержку MySQL осуществляет корпорация Oracle, получившая права на торговую марку вместе с поглощённой Sun Microsystems, которая ранее приобрела шведскую компанию MySQL AB. Продукт распространяется как под GNU General Public License, так и под собственной коммерческой лицензией. Помимо этого, разработчики создают функциональность по заказу лицензионных пользователей.\cite{mysql}\\
\subsection{СУБД SQLite}
SQLite — это встраиваемая кроссплатформенная БД, которая поддерживает достаточно полный набор команд SQL и доступна в исходных кодах (на языке C). Исходные коды SQLite находятся в public domain, то есть вообще никаких ограничений на использование.\\
\subsection{Обоснование выбора СУБД}
Для наибольшего удобства была использована SQLite, что позволило достаточно сильно ускорить скорость открытия потребления, а также существенно снизить потребляемое количество оперативной памяти. В ранних версиях программы при наличии большого количества записей в базе данных программа пыталась полностью перенести данные в память, что приводило к общему замедлению работы операционной системы и возможному повреждению данных.

\section{Обзор и выбор языка программирования}
При выборе языка программирования для создания приложения я остановился на трех базовых концепциях:

\begin{enumerate}
	\item Портируемость
	\item Легкость
	\item Удобство использования
	\item Большие возможности
\end{enumerate}

Благодаря этому списку мне удалось достаточно сильно сократить число
языков программирования на которых можно было написать приложение, в
итоге я остановился на языке

C++ с расширением QT Framework. Который оказался достаточно комфортным и легко расширяемым для выполнения необходимых задач.

\part{Разработка базы данных для хранения информации}
\section{Проектирование базы данных}
База данных состоит из 1 таблицы имеющую создающуюся с помощью следующего запроса:
\begin{MyCode}
	CREATE TABLE  IF NOT EXISTS "operations"(
	"id" INTEGER  PRIMARY KEY  NOT NULL ,
	"time" DATETIME DEFAULT (CURRENT_TIMESTAMP),
	"summ" NOT NULL  DEFAULT (null),
	"comment"  NOT NULL,
	"catid" INTEGER DEFAULT (null),
	"side" BOOL DEFAULT (0))
\end{MyCode}

\section{Реализация базы данных в СУБД SQLite}
Таблица создается в базе данных при запуске программы используя запрос указанный выше\\

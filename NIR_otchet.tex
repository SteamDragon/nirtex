%
% Шаблон для НИР
%

\documentclass[a4paper,12pt]{article}
\usepackage[backend=biber,sorting=none]{biblatex} % библиография
\usepackage{mathtext} %русские буквы в формулах
\usepackage[T2A]{fontenc}
\usepackage[utf8]{inputenc}
\usepackage[russian]{babel}
\usepackage{amsmath}
\usepackage{fancyvrb}
\usepackage{formular}
\usepackage{setspace} % управление междустрочными интервалами
%поля документа
\usepackage[left=3cm,right=1cm,top=2cm,bottom=2cm]{geometry}

\usepackage{misccorr} % точки в конце номеров разделов, использовать перед пакетом ccaption!
\usepackage{ccaption} % изменения подписей к рисункам и табл.
% отступ перед первым абзацем
\usepackage{indentfirst}
%вставка изображений
\usepackage{graphicx}
% счетчики
\usepackage{totcount}
% управление содержанием
\usepackage{tocloft}
% управление таблицами и рисунками
\usepackage{float}
\usepackage{tabularx}

%для добавления количества источников в реферат
\newtotcounter{citnum} %From the package documentation
\def\oldbibitem{} \let\oldbibitem=\bibitem
\def\bibitem{\stepcounter{citnum}\oldbibitem}

% окружение для листингов - с нумерацией строк слева
\DefineVerbatimEnvironment{MyCode}{Verbatim}{frame=lines,numbers=left,numberblanklines=false,framesep=5mm}

% автоматическая нумерация листингов
\newfloat{Program}{phb}{lop}
\floatname{Program}{Листинг}
\floatstyle{ruled}

\setcounter{secnumdepth}{3} % глубина нумерации до подразделов

%если нужны точки в оглавлении для разделов - раскомментируйте следующую команду
%\renewcommand{\cftsecleader}{\cftdotfill{\cftdotsep}}

\addto\captionsrussian{
\renewcommand{\figurename}{Рисунок}
\renewcommand{\tablename}{Таблица}
}

% дефис в подписи к рисункам
\captiondelim{ -- } 

% Настройки для окружений с подчеркиваниями для подписей и пр.
\setFRMfontencoding{T2A}
\setFRMdfontencoding{T2A}
% thanks to A.Starikov
\setFRMfontfamily{cmr}
\setFRMdfontfamily{ptm}
\setFRMdfontsize{10pt}

% задает длину поля для подписи на титульной странице
\newFRMfield{xtitlesign}{70mm}

% поле для факультета или кафедры
\newFRMfield{fcath}{65mm}


\addbibresource{rbiblio.bib}

\begin{document}

% счетчики страниц, рисунков, таблиц
\regtotcounter{page}
\regtotcounter{figure}
\regtotcounter{table}

\renewcommand{\refname}{\centerline{СПИСОК ИСПОЛЬЗОВАННОЙ ЛИТЕРАТУРЫ}} 
\renewcommand{\contentsname}{\centerline{СОДЕРЖАНИЕ}} 
%\renewcommand{\refname}{Список источников}  % По умолчанию "Список литературы" (article)
%\renewcommand{\bibname}{Литература}  % По умолчанию "Литература" (book и report)

% титульная страница
\thispagestyle{empty}
\begin{center} \small
 
\textbf{МИНИСТЕРСТВО НАУКИ И ВЫСШЕГО ОБРАЗОВАНИЯ РОССИЙСКОЙ ФЕДЕРАЦИИ}\\
ФЕДЕРАЛЬНОЕ ГОСУДАРСТВЕННОЕ АВТОНОМНОЕ ОБРАЗОВАТЕЛЬНОЕ УЧРЕЖДЕНИЕ ВЫСШЕГО ОБРАЗОВАНИЯ\\
«Национальный исследовательский ядерный университет «МИФИ»\\
\textbf{Обнинский институт атомной энергетики} – \\
филиал федерального государственного автономного образовательного учреждения высшего\\
образования «Национальный исследовательский ядерный университет «МИФИ»\\
(ИАТЭ НИЯУ МИФИ)
Отделение Интеллектуальные кибернетические системы
\end{center}

\begin{center}
\textbf{\Large Выпускная квалификационная работа -\\ бакалаврская работа} \\
	по направлению подготовки: 09.03.02 Информационные системы и технологии\\	
	Направленность (профиль): Информационные технологии
	\medskip
	
	\textbf{\Large 
		Построение системы учета семейного бюджета с использованием QT Framework
	}
	
\end{center}

\vspace{1cm}

\setlength\tabcolsep{0pt}
\setlength\thickmuskip{0mu}
\setlength\medmuskip{0mu}
\begin{tabular*}{50px}{lcr}
Выполнил: & & \\
Студент гр. ИС-Б14 & \useFRMfield{xtitlesign} & Миронов А.В.\\
& \small(подпись, дата) & \\
Руководитель ВКР, & & \\
доцент отделения ИКС, к.т.н. & \useFRMfield{xtitlesign} & Мирзеабасов О.А.\\
& \small(подпись, дата) & \\
Консультант по ВКР,  & & \\
доцент отделения ИКС, к.т.н. & \useFRMfield{xtitlesign} & Грицюк С.В.\\
& \small(подпись, дата) & \\
Нормоконтроль& \useFRMfield{xtitlesign} & Бязрова В.Ф.\\
& \small(подпись, дата) & \\
Выпускная квалификационная & &\\
работа допущена к защите & \useFRMfield{xtitlesign} &\\
& \small(№ протокола, дата заседания комиссии) & \\
Руководитель & & \\
образовательной программы & & \\
09.03.02 Информационные & & \\
системы и технологии & & \\
доцент отделения ИКС, к.т.н. & \useFRMfield{xtitlesign} & Мирзеабасов О.А.\\
& \small(подпись, дата) & \\
\end{tabular*}


\vfill
\large

\begin{center}
Обнинск, 2019 г
\end{center}

\onehalfspacing

\pagebreak

% реферат 
\thispagestyle{empty}

\section*{\centering РЕФЕРАТ}

% возможно, кол-во источников придется вставлять вручную
\total{page} стр., \total{table} табл., \total{figure} рис. , \total{citnum} ист. 

\pagebreak
\thispagestyle{empty}


\section*{\centering ОБОЗНАЧЕНИЯ И СОКРАЩЕНИЯ}


\pagebreak



\tableofcontents
% если нужно добавить "Стр." над номерами страниц - раскомментируйте следующую команду
%\addtocontents{toc}{~\hfill\textbf{Стр.}\par}

\pagebreak

\section*{\centering ВВЕДЕНИЕ}
\addcontentsline{toc}{section}{ВВЕДЕНИЕ}


Обработка данных с помощью различных языков программирования приобретает всё большую популярность с каждым годом. Одним из языков для статистической обработки и визуализации данных является \textbf{R}. В языке \textbf{R} существуют пакеты для визуализации и вычисления траекторий динамических систем. 

Задачи, решаемые в ходе работы (в соответствии с заданием на НИР):

\begin{enumerate}
    \item что-то сначала
    \item что-то потом
    \item подготовка отчета. 
	
\end{enumerate}
 % текст введения в файле intro.tex
\pagebreak

%\input{Post_zad}
\pagebreak
% первая глава

\section{Название первого раздела}

\subsection{Таблицы}

Текст подраздела посвящен таблице~\ref{tab:t1}

\begin{table}[H]
\caption{Пример таблицы}
\label{tab:t1}
\begin{center}
\begin{tabular}{|r|p{5.5cm}|p{2.5cm}|}
\hline 
1 & Первый текст в ячейке фиксированной ширины & $E=mc^2 $ \\ 
\hline 
12 & Второй текст тоже может быть произвольно длинным & $\sin \pi = 0 $ \\ 
\hline 
\end{tabular} 
\end{center}
\end{table}

\subsection{Вставка изображения}

\begin{figure}[H]
	\centering
	\includegraphics[width=0.7\linewidth]{pics/pic3D}
	\caption{Пример изображения для демонстрации возможности вставки в документ}
	\label{fig:pic3d}
\end{figure}



Изображение на рис.~\ref{fig:pic3d} находится в подкаталоге pics.

  % первая глава - в файле part1.tex
\pagebreak
% второй раздел - файл part2.tex

\section{Методики учета семейного бюджета}

\subsection{Где вести учет семейного бюджета}

\subsubsection{Тетрадь или амбарная книга}
Несомненно, что для вычислений, связанных с учетом личных финансов,
было бы удобно воспользоваться компьютером и вести в нем все записи, однако если такой возможности нет, то можно завести тетрадь или амбарную
книгу. В самом простом обобщенном случае рекомендуется разбить лист на
три графы:
\begin{table}[H]
\caption{Пример теблицы для учета семейного бюджета}
\label{tab:t1}
\begin{center}
\begin{tabular}{|r|p{5.5cm}|p{2.5cm}|}
\hline 
Доход & Расход & Итого \\ 
\hline 
 &  &  \\ 
\hline 
\end{tabular} 
\end{center}
\end{table}

Графы Расход и Доход будут отражать соответствующее движение
денег вашего кошелька, а графа Итого нужна для того, чтобы сверять цифры на бумаге с количеством денег в карманах. Как ни странно, они должны
совпадать.
Такой подход в целом приемлем для одного человека, он даже позволит
отследить и выявить необязательные расходы, которые впоследствии можно
уменьшить или вовсе убрать. Однако в таком виде о какой-либо наглядности
и систематизации говорить не приходится. Тем более в рамках рассмотрения
бюджета семьи. Ведь, как уже говорилось в предыдущей части, семейный
бюджет охватывает множество составляющих.
Для повышения наглядности и хоть какой-то систематизации доходов и
расходов, приведенную табличку необходимо разбавитьї дополнительными
колонками группируя разные виды расходов в соответствии реально имеющимся.
Например, в первую колонку можно записывать коммунальные платежи,
свет, интернет или аренду. Во второй колонке записывать лишь траты на
продукты питания, в третьей личные расходы, в четвертой расходы на развлечения и в пятой непредвиденные расходы.
Естественно, существующую таблицу нужно модернизировать под себя и
вероятно кто-то посчитает нужным добавить колонки по бытовой химии, уходу за кошкой, ребенком, родителями и т.д.
Эти расширения, в конце концов, приведут к тому, что таблица попросту
перестанет умещаться даже в амбарную книгу. И в этом случае на помощь
приходят компьютерные программные средства.

\subsubsection{Электронные таблицы}
Более продвинутый путь, приступить к ведению семейного бюджета при
помощи электронной таблицы (Excel, Google Docs и т.п.), где уже даже имеются основные формулы для анализа бюджета. По сути, вам остается лишь
выбрать и применить их к своим данным. Но и тут есть путь проще.
Дело в том, что на сегодняшний день существует множество специальных
шаблонов для электронных таблиц, в которых уже учтены некоторые наиболее популярные поля и необходимые для расчетов формулы.
\subsubsection{Специализированные программы}
Кроме шаблонов для табличных редакторов, в сети интернет предлагается
масса специальных программ для ведения учета и планирования семейного
бюджета.
Они позволяют автоматизировать большую часть работы, что значительно
упрощает процесс ведения домашних финансов.
Помимо того, эти программы, как правило, имеют массу вспомогательных функций, которые позволяют выявить слабые и сильные стороны вашего отношения с деньгами, помогут явно обратить внимание на, казалось бы,
очевидные, вещи, но почему-то не используемые в повседневной жизни. По
сути, программы для ведения семейного бюджета значительно облегчают и
помогают создать целостную картину наших взаимоотношений с финансами.
 % вторая глава - в файле part2.tex
\pagebreak
% !TeX spellcheck = de_DE
\part{Выбор инструментов и средств разработки}
\section{Обзор и выбор СУБД}
\subsection{СУБД MySQL}
MySQL — свободная реляционная система управления базами данных. Разработку и поддержку MySQL осуществляет корпорация Oracle, получившая права на торговую марку вместе с поглощённой Sun Microsystems, которая ранее приобрела шведскую компанию MySQL AB. Продукт распространяется как под GNU General Public License, так и под собственной коммерческой лицензией. Помимо этого, разработчики создают функциональность по заказу лицензионных пользователей.\cite{mysql}\\
\subsection{СУБД SQLite}
SQLite — это встраиваемая кроссплатформенная БД, которая поддерживает достаточно полный набор команд SQL и доступна в исходных кодах (на языке C). Исходные коды SQLite находятся в public domain, то есть вообще никаких ограничений на использование.\\
\subsection{Обоснование выбора СУБД}
Для наибольшего удобства была использована SQLite, что позволило достаточно сильно ускорить скорость открытия потребления, а также существенно снизить потребляемое количество оперативной памяти. В ранних версиях программы при наличии большого количества записей в базе данных программа пыталась полностью перенести данные в память, что приводило к общему замедлению работы операционной системы и возможному повреждению данных.

\section{Обзор и выбор языка программирования}
При выборе языка программирования для создания приложения я остановился на трех базовых концепциях:

\begin{enumerate}
	\item Портируемость
	\item Легкость
	\item Удобство использования
	\item Большие возможности
\end{enumerate}

Благодаря этому списку мне удалось достаточно сильно сократить число
языков программирования на которых можно было написать приложение, в
итоге я остановился на языке

C++ с расширением QT Framework. Который оказался достаточно комфортным и легко расширяемым для выполнения необходимых задач.

\part{Разработка базы данных для хранения информации}
\section{Проектирование базы данных}
База данных состоит из 1 таблицы имеющую создающуюся с помощью следующего запроса:
\begin{MyCode}
	CREATE TABLE  IF NOT EXISTS "operations"(
	"id" INTEGER  PRIMARY KEY  NOT NULL ,
	"time" DATETIME DEFAULT (CURRENT_TIMESTAMP),
	"summ" NOT NULL  DEFAULT (null),
	"comment"  NOT NULL,
	"catid" INTEGER DEFAULT (null),
	"side" BOOL DEFAULT (0))
\end{MyCode}

\section{Реализация базы данных в СУБД SQLite}
Таблица создается в базе данных при запуске программы используя запрос указанный выше\\
 % вторая глава - в файле part3.tex


\pagebreak
\section*{\centering ЗАКЛЮЧЕНИЕ}
\addcontentsline{toc}{section}{ЗАКЛЮЧЕНИЕ}
Настоящая научно-исследовательская работа посвящена созданию приложения по ведению домашней бухгалтерии на языке

C++ с расширением QT

Framework. В результате была разработано приложение, позволяющее учитывать расходы и доходы в достаточно удобном интерфейсе.
% оформление библиографии - вариант с БД

\pagebreak
\nocite{qtdoc}
\nocite{qt532015}
\nocite{Warren2007}
\addcontentsline{toc}{section}{СПИСОК ИСПОЛЬЗОВАННОЙ ЛИТЕРАТУРЫ}
\printbibliography

\pagebreak
\section*{\centering Приложение}
\addcontentsline{toc}{section}{Приложение}


\end{document}          

